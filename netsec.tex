\documentclass{beamer}

\usepackage{graphicx}
\usepackage{hyperref}

\usetheme{metropolis}

\title{Seed Heartbleed Lab}
\date{\today}
\author{Ward M. Zahran}
\institute{University of Jordan}
\begin{document}
\maketitle
\section{Prelude}

\begin{frame}{License and Source}
    This work is licensed under CC BY-SA 4.0

    Source files and images can be found on:
    \href{https://github.com/ward-zahran/netsec-project-presentation/}{https://github.com/ward-zahran/netsec-project-presentation/}


\end{frame}

\begin{frame}{The Why}
    From RFC 6520

    ``Sending HeartbeatRequest messages allows the sender to make sure that it can reach the peer and the peer is alive.''

\end{frame}

\begin{frame}{The Why (cont.)}
    It is mostly meant to be used with unreliable protocols which have no session management
    to keep the connection alive without continuous data transfer.
\end{frame}

\begin{frame}{The How}
    The protcol consists of two types of messages:
    \begin{itemize}
        \item HeartbeatRequest
        \item HeartbeatResponse
    \end{itemize}
\end{frame}


\begin{frame}[fragile]{Heartbeat Message Structure}
    \begin{verbatim}
struct
{
    HeartbeatMessageType type;
    uint16 payload_length;
    opaque payload[HeartbeatMessage.payload_length];
    opaque padding[padding_length];
}
HeartbeatMessage;

    \end{verbatim}


\end{frame}

\begin{frame}{Heartbeat Message Structure Explained}
    \begin{itemize}
        \item

            type:  The message type, either heartbeat\_request or
            heartbeat\_response.


        \item
            payload\_length:  The length of the payload.

        \item
            payload:  The payload consists of arbitrary content.


        \item
            padding:  The padding is random content that MUST be ignored by the
            receiver.
    \end{itemize}

\end{frame}

\begin{frame}{The How (cont.)}

    Note that both the request and response are the same just with a different type field value.

    A peer sends a HeartbeatRequest and then if the other peer is still alive it should echo
    back the payload in a HeartbeatResponse.

\end{frame}

\section{Heartbleed}
\begin{frame}{Heartbleed Explained}
    Heartbleed is a vulnerability in the OpenSSL implementation of the Heartbeat protocol.

\end{frame}

\begin{frame}{Heartbleed Explained (cont.)}
    It results in the ability to leak an applications memory by giving a larger payload\_size
    than the actual payload in the HeartbeatRequest, the OpenSSL implementation didn't have
    bound checking code resulting in a buffer over-read vulnerability, which is down to a
    memcpy() statement which uses the unchecked payload\_length variable to copy the
    contents of the payload from the servers memory into the response.

\end{frame}

\begin{frame}{Heartbleed Visualization}
    % Credit for image: https://en.wikipedia.org/wiki/File:Simplified_Heartbleed_explanation.svg
    % By: FenixFeather
    \includegraphics[width=\textheight]{heartbleed_visualization.png}
\end{frame}

\section{Demo}
\begin{frame}{Demo Backround}
    The goal of this demo is to extract secrets from ELGG (a self-hosted social media)
    instance using a vulnerable version of OpenSSL.

\end{frame}

\begin{frame}{Demo Background (cont.)}
    The first step is logging into the instance's admin account and interacting with a user
    so we have data to test on, then we run the heartbleed
    attack script on the attacker machine to leak the server memory.
\end{frame}

\begin{frame}{Sending A Message}
    \includegraphics[width=\textwidth]{sent_message.png}
\end{frame}

\begin{frame}{Leaked Credentials}
    \includegraphics[width=\textwidth]{leaked_credentials.png}
\end{frame}

\begin{frame}{Logging to Account from the Attacker}
    \includegraphics[width=\textwidth]{attacker_sent_message.png}
\end{frame}

\begin{frame}{Countermeasures}
    The Heartbleed vulnerability can be patched by updating to version 1.0.1g or newer of
    OpenSSL.
\end{frame}

\section{Misc.}
\begin{frame}{payload\_length and Its Effect}
    \begin{itemize}
        \item
            The payload length is an unsigned 16 bit integer meaning we have a range of 0 - 65535
            to select from, but using a payload\_length of 22 or lower he Heartbeat query will
            receive a response packet without attaching any extra data making it benign.
        \item
            Obviously the side effect of increasing the payload\_length will result in more data
            sent back to the attacker.
    \end{itemize}
\end{frame}

\begin{frame}{Heartbeat payload\_length Boundry}
    \includegraphics[width=\textwidth]{heartbleed_boundry.png}
\end{frame}

\begin{frame}
    Thanks for listening.
\end{frame}
\end{document}

